\documentclass[10pt,-letter paper]{article}
\usepackage[left=1in, right=0.75in, top=1in, bottom=0.75in]{geometry}
\usepackage{graphicx} % Required for inserting images
\usepackage{siunitx}
\usepackage{setspace}
\usepackage{gensymb}
\usepackage{xcolor}
\usepackage{caption}
%\usepackage{subcaption}
\doublespacing
\singlespacing
\usepackage[none]{hyphenat}
\usepackage{amssymb}
\usepackage{relsize}
\usepackage[cmex10]{amsmath}
\usepackage{mathtools}
\usepackage{amsmath}
\usepackage{commath}
\usepackage{amsthm}
\interdisplaylinepenalty=2500
%\savesymbol{iint}
\usepackage{txfonts}
%\restoresymbol{TXF}{iint}
\usepackage{wasysym}
\usepackage{amsthm}
\usepackage{mathrsfs}
\usepackage{txfonts}
\let\vec\mathbf{}
\usepackage{stfloats}
\usepackage{float}
\usepackage{cite}
\usepackage{cases}
\usepackage{subfig}
%\usepackage{xtab}
\usepackage{longtable}
\usepackage{multirow}
%\usepackage{algorithm}
\usepackage{amssymb}
%\usepackage{algpseudocode}
\usepackage{enumitem}
\usepackage{mathtools}
%\usepackage{eenrc}
%\usepackage[framemethod=tikz]{mdframed}
\usepackage{listings}
%\usepackage{listings}
\usepackage[latin1]{inputenc}
%%\usepackage{color}{   
%%\usepackage{lscape}
\usepackage{textcomp}
\usepackage{titling}
\usepackage{hyperref}
%\usepackage{fulbigskip}   
\usepackage{tikz}
\usepackage{graphicx}
\lstset{
  frame=single,
  breaklines=true
}
\let\vec\mathbf{}
\usepackage{enumitem}
\usepackage{graphicx}
\usepackage{siunitx}
\let\vec\mathbf{}
\usepackage{enumitem}
\usepackage{graphicx}
\usepackage{enumitem}
\usepackage{tfrupee}
\usepackage{amsmath}
\usepackage{amssymb}
\usepackage{mwe} % for blindtext and example-image-a in example
\usepackage{wrapfig}
\graphicspath{{figs/}}
\providecommand{\cbrak}[1]{\ensuremath{\left\{#1\right\}}}
\providecommand{\brak}[1]{\ensuremath{\left(#1\right)}}
\newcommand{\sgn}{\mathop{\mathrm{sgn}}}
\providecommand{\abs}[1]{\left\vert#1\right\vert}
\providecommand{\res}[1]{\Res\displaylimits_{#1}} 
\providecommand{\norm}[1]{\left\lVert#1\right\rVert}
%\providecommand{\norm}[1]{\lVert#1\rVert}
\providecommand{\mtx}[1]{\mathbf{#1}}
\providecommand{\mean}[1]{E\left[ #1 \right]}
\providecommand{\fourier}{\overset{\mathcal{F}}{ \rightleftharpoons}}
%\providecommand{\hilbert}{\overset{\mathcal{H}}{ \rightleftharpoons}}
\providecommand{\system}{\overset{\mathcal{H}}{ \longleftrightarrow}}
 %\newcommand{\solution}[2]{\textbf{Solution:}{#1}}
%\newcommand{\solution}{\noindent \textbf{Solution: }}
\newcommand{\cosec}{\,\text{cosec}\,}
\providecommand{\dec}[2]{\ensuremath{\overset{#1}{\underset{#2}{\gtrless}}}}
\newcommand{\myvec}[1]{\ensuremath{\begin{pmatrix}#1\end{pmatrix}}}
\newcommand{\myaugvec}[2]{\ensuremath{\begin{amatrix}{#1}#2\end{amatrix}}}
\newcommand{\mydet}[1]{\ensuremath{\begin{vmatrix}#1\end{vmatrix}}}
\title{MATHEMATICS}
\author{SECTION A}
\date{\today}
\begin{document}

\maketitle

\begin{enumerate}
\section{Matrix}
\item If \myvec{A} is a square matrix of order $3$ with $\mydet{A} = 4$, then write the value of $\mydet{-2A}$.
\item If $A=\myvec{ -3 & 6 \\ -2 & 4}$, then show that ${A}^3=A$.
\item Using properties of determinants, prove that
	\begin{align*}
\myvec{{a}^2 + {1} & {a}{b} & {a}{c} \\ {a}{b} & {b}^2+{1} & {b}{c} \\ {a}{c} & {b}{c} & {c}^2+{1}} = 1+{a}^2+{b}^2+{c}^2
	\end{align*}
\item If $A=\myvec{1&-1&1 \\ 2&-1&0 \\ 1&0&0}$, find ${A}^2$ and show that ${A}^2 = A^{-1}$.
\item Using matrix method, solve the following system of equations:
	\begin{align*}
		2{x}-3{y}+5{z}=13 \\
		3{x}+2{y}-4{z}=-2 \\
		{x}+{y}-2{z}=-2
	\end{align*}
\section{Differentiation}
\item Find the integrating factor of the differential equation ${x}\dfrac{dy}{dx}-2{y} = 2{x}^2$.
\item Find $\dfrac{dy}{dx}$, if ${x}{y}^2-{x}^2$ = $4$.
\item Form the differential equation representing the family of curves ${y}^2 = m\brak{{a}^2-{x}^2}$ by eliminating the arbitrary constants $'m'$ and $'a'$.
\item If $\sin{y} = x\sin{\brak{a + y}}$, prove that 
	\begin{align*}
		\dfrac{dy}{dx} = \frac{\sin^{2}\brak{a + y}}{\sin{a}}
	\end{align*}
\item If $\brak{\sin x}^y = {x + y}$, find $\dfrac{dy}{dx}$.
\item If $y = \brak{\cot^{-1}{x}}^2$, show that $\brak{{x}^2+{1}}^2\dfrac{d^2y}{dx^2} + 2{x}\brak{{x^2}+{1}}\dfrac{dy}{dx} = 2$.
\section{Integration}
\item Find: 
	\begin{align*}
		\int\frac{\sin x-\cos x}{\sqrt{1+\sin 2x}}dx, 0<x<\frac{\pi}{2}
	\end{align*}
\item Find: 
	\begin{align*}
		\int\frac{\sin\brak{x-a}}{\sin\brak{x+a}}dx
	\end{align*}
\item Find:
	\begin{align*}
	\int\brak{\log{x}}^2dx
        \end{align*}
\section{Probability}
\item Mother, father and son line up at random for a family photo. If A and B are two events given by A = son on one end, B = Father in the middle, find $P(B\mid A)$.
\item Let ${X}$ be a random variable which assumes values ${x_1},{x_2},{x_3},{x_4}$ such that
	\begin{align*}
2P\brak{X = x_1} = 3P\brak{X = x_2} = P\brak{X = x_3} = 5P\brak{X = x_4}.
	\end{align*}
Find the probability distribution of ${X}$.
\item A coin is tossed 5 times. Find the probability of getting 
	\begin{enumerate}[label = (\roman*)]
		\item at least 4 heads, and 
		\item at most 4 heads.
	\end{enumerate}
\section{Vectors}
\item If a line has the direction ratios ${-18,12,-4}$, then what are its direction cosines?
\item Find the cartesian equation of the line which passes through the point $\brak{-2,4,-5}$ and is parallel to the line $\frac{{x} + {3}}{3} = \frac{{4} - {y}}{5} = \frac{{z} + {8}}{6}$.
\item Find a unit vector perpendicular to both the vectors $\overrightarrow{a}$ and $\overrightarrow{b}$, where $\overrightarrow{a} = \hat{i} - 7\hat{j} + 7\hat{k} and \overrightarrow{b} = 3\hat{i} - 2\hat{j} + 2\hat{k}$.
\item Show that the vectors $\hat{i}-2\hat{j} +3\hat{k}$, $-2\hat{i}+3\hat{j}-4\hat{k}$ and $\hat{i}-3\hat{j}+5\hat{k}$ are coplanar.
\section{Functions}
\item Let * be a binary operation on $\vec{R}-{-1}$ defined by $a * b = \frac{a}{{b} + {1}}$, for all $a,b \in \vec{R}-{-1}$. show that * is neither commutative nor associative in $\vec{R}-{-1}$.
\item Show that the relation $\vec{R}$ on the set $\vec{Z}$ of all integers, given by $R = \cbrak{\brak{a,b} :2 \text{divides} \brak{a-b}}$ is an equivalence relation.
\item If $f\brak{x} = \frac{4{x}+3}{6{x}-4}$, ${x}\neq\frac{2}{3}$, show that $fof\brak{x} = {x}$ for all ${x}\neq\frac{2}{3}$. Also, find the inverse of ${f}$.
\item Find the local maxima and local minima, if any, of the following function. Also find the local maximum and the local minimum values, as the case may be : 
	\begin{align*}
		f\brak{x} = \sin{x}+\frac{1}{2} \cos{2x}, 0\leq{x}\leq \frac{\pi}{2}
	\end{align*}
\section{Algebra}
\item If $\sin^{-1}\brak{\frac{3}{x}} + \sin^{-1}\brak{\frac{4}{x}} = \frac{\pi}{2}$, then find the value of $x$.
\section{Intersection of Conics}
\item Find the equations of the tangent and normal to the curve $y = \frac{x-7}{\brak{x-2}\brak{x-3}}$ at the point where it cuts the x-axis.
	\end{enumerate}
	\end{document}
